\section*{The Inverse Fourier Transform and Spectral Processing}

\subsection*{The Inverse Discrete Fourier Transform}

We have seen how a time domain signal can be projected onto a basis of cosine and sine waves to give the
frequency domain representation of the sound.  Is it possible to inverse that process?  Can we take a signal
in the frequency domain and convert it to the time domain?  Fortunately, the answer is yes!  This technique
is called the Inverse Discrete Fourier Transform (abbreviated as IDFT).  Equation \ref{eq:idft} shows the equation
for the IDFT.  

\begin{equation}
	\label{eq:idft}
	x[n] = \frac{1}{N}\sum_{k = 0}^{N - 1}X_k \cdot e^{i\frac{2\pi}{N}kn}
\end{equation}

If we examine the IDFT, we can see that we sum the inner product of a complex sinusoids against all of our 
complex frequency bins $X_k$ to restore the original signal $x[n]$.  The equation for the IDFT is very similar
to the one for the DFT shown in Equation \ref{eq:dft}.  The only differences are $x[n]$ and $X_k$ are swapped,
the complex exponential is now positive, and the IDFT has a scaling constant of $\frac{1}{N}$.  The appendix
shows a proof that the IDFT and DFT are truly inverse operations.  The beauty of the IDFT is that it allows us
to convert back and forth between the frequency domain and the time domain.  We can convert a signal to the
frequency domain, perform some spectral manipulations and then convert back to the time domain, creating
a processed version of the original.  Though less frequently done, we could also build a signal in the frequency
domain and then use the IDFT to get the time domain representation of the sound.  At the end of the day, all 
signals need to be converted back to the time domain because all digital-to-analog converters (ADCs) require
time domain representations of sound.

\subsection*{Spectral Processing}

The DFT and IDFT provide the means to convert an audio signal to the frequency domain for processing and
then convert back to the time domain.  Spectral manipulation involves changing the complex numbers for
each $X_k$.  Simple operations include zeroing out bins, scaling the magnitude, changing the phase, rearranging
bins, etc.  There is a host of different techniques that one could apply.  One seemingly obvious example is
to create a brickwall filter by zeroing out frequency bins to create a high-pass, low-pass, or bandpass filter.
Zeroing out bins only creates a brickwall filter if the signal $x[n]$ is periodic along $N$.  Otherwise, nasty artifacts
will be present in the time domain once the IDFT is complete.  Discussing and analyzing various spectral 
techniques is beyond the scope of this guide; however, experimenting with simple spectral operations is good 
practice for developing intuitions.